Assignment 2 - Nina Bartmann

Course: Economics and Psychology of Social Norms and Strategic Behavior

$$\alpha = \beta = \frac{1}{2}$$
Total Utility for player 2: 
$$U_2(u_1, u_2) = w-$\frac{1}{2}$e - $\frac{1}{2}$ max\Big\{2e-w-w + $\frac{1}{2}$e ; 0\Big\}- $\frac{1}{2}$ max\Big\{w-$\frac{1}{2}$e -2e+w; 0\Big\}$$
$$\Rightarrow U_2(u_1, u_2) = w-$\frac{1}{2}$e - $\frac{1}{2}$ max\Big\{$\frac{5}{2}$e-2w ; 0\Big\}- $\frac{1}{2}$ max\Big\{2w-$\frac{5}{2}$e ; 0\Big\}$$
Player 2 maximizes total utility by minimizing: $\frac{1}{2}$ max\Big\{$\frac{5}{2}$e-2w ; 0\Big\}+ $\frac{1}{2}$ max\Big\{2w -\frac{5}{2}e ; 0\Big\}$
$$Solution: $\frac{5}{2}$e = 2w and e = $\frac{4}{5}$w$$
Player 2 will set his effort level at $\frac{4}{5}$ of his wage. 

Total Utility for player 1: 
$$U_1(u_1, u_2) = 2e - w ? $\frac{1}{2}$ max\Big\{w - $\frac{1}{2}$e -2e+w ; 0\Big\} - $\frac{1}{2}$ max\Big\{2e-w-w +$\frac{1}{2}$e; 0\Big\}$$
Player 1 can expect that player 2 will set his effort level $e = \frac{4}{5}w$
$$U1(u1, u2) = $2\times\frac{4}{5}$w - w - $\frac{1}{2}$ max\Big\{0 ; 0\Big\} - $\frac{1}{2}$ max\Big\{0; 0\Big\} = $\frac{3}{5}$w$$
Player 1's total utility increases with wage, therefore, he will choose the maximum level that he can choose for wage in order to maximize his utility. He will choose w = 1

Inequality Aversion (Bolton-Ockenfels) 
$$U(A) = 10 - \theta_2(0.28 - \frac{1}{3})^2$$
$$U(B) = 10 - \theta_2(0.3 - \frac{1}{3})^2$$
$$U(B) = 10 - \theta_2(0.33 - \frac{1}{3})^2$$
$$\Rightarrow Regardless of the measure of inequality aversion \theta_2, \theta_2(0.33 - \frac{1}{3}) is always smaller compared to allocations under A or B. Hence, utility under C is highest. 

$$y = \Bigg(x - 1.5 \times \bigg(frac{50 + x}{2}\bigg)\Bigg)^2$$
Make a graph!

Determining the Quantal Response Equilibrium (Section 6.2.5 of Cartwright):
$$q_u = \frac{e^(x\times\lambda\times p)}{e^(x\times\lambda\timesp) + e^(\lambda(1 - q))}$$
$$p_L = \frac{e^(\lambda\times q)}{e^(\lambda\times q) + e^(\lambda(1 - q))}$$

Game Theory 1: Assignment 1
Question: Derive the Nash equilibrium in pure strategies if the frms choose their quantities simultaneously. What happens if n approaches infinity??
Solution: Each firm i maximizes profits $\pi_i = q_i(P - c) = q_i(a - c - (q_i + \sum_{j\neqi} q_j))$ 
FOC: 
$$\frac{\partial\pi_i}{\partial q_i} = a - c - 2q_i - \sum_{j\neqi} q_j = 0$$
Because of symmetry we have: $q_j = q_i$ for all $j \neq i$ in equilibrium. Hence $q_i* = \frac{a-c}{n+1}$, which is strictly decreasing in n. 
For $n \rightarrow \infty$ we have $q_i* \rightarrow 0$, i.e. each firm captures an ever decreasing market share (and profits). Since $Q* = nq_i* = \frac{n}{n+1} (a - c)$, which is increasing in n, total output increases. Hence, P decreases and approaches the limit result P* = a - Q* = a - (a - c) = c.  
